% Options for packages loaded elsewhere
\PassOptionsToPackage{unicode}{hyperref}
\PassOptionsToPackage{hyphens}{url}
%
\documentclass[
  10pt,
]{article}
\usepackage{amsmath,amssymb}
\usepackage{iftex}
\ifPDFTeX
  \usepackage[T1]{fontenc}
  \usepackage[utf8]{inputenc}
  \usepackage{textcomp} % provide euro and other symbols
\else % if luatex or xetex
  \usepackage{unicode-math} % this also loads fontspec
  \defaultfontfeatures{Scale=MatchLowercase}
  \defaultfontfeatures[\rmfamily]{Ligatures=TeX,Scale=1}
\fi
\usepackage{lmodern}
\ifPDFTeX\else
  % xetex/luatex font selection
\fi
% Use upquote if available, for straight quotes in verbatim environments
\IfFileExists{upquote.sty}{\usepackage{upquote}}{}
\IfFileExists{microtype.sty}{% use microtype if available
  \usepackage[]{microtype}
  \UseMicrotypeSet[protrusion]{basicmath} % disable protrusion for tt fonts
}{}
\makeatletter
\@ifundefined{KOMAClassName}{% if non-KOMA class
  \IfFileExists{parskip.sty}{%
    \usepackage{parskip}
  }{% else
    \setlength{\parindent}{0pt}
    \setlength{\parskip}{6pt plus 2pt minus 1pt}}
}{% if KOMA class
  \KOMAoptions{parskip=half}}
\makeatother
\usepackage{xcolor}
\usepackage[margin=1cm, paperwidth=21cm, paperheight=14.8cm]{geometry}
\usepackage{graphicx}
\makeatletter
\def\maxwidth{\ifdim\Gin@nat@width>\linewidth\linewidth\else\Gin@nat@width\fi}
\def\maxheight{\ifdim\Gin@nat@height>\textheight\textheight\else\Gin@nat@height\fi}
\makeatother
% Scale images if necessary, so that they will not overflow the page
% margins by default, and it is still possible to overwrite the defaults
% using explicit options in \includegraphics[width, height, ...]{}
\setkeys{Gin}{width=\maxwidth,height=\maxheight,keepaspectratio}
% Set default figure placement to htbp
\makeatletter
\def\fps@figure{htbp}
\makeatother
\setlength{\emergencystretch}{3em} % prevent overfull lines
\providecommand{\tightlist}{%
  \setlength{\itemsep}{0pt}\setlength{\parskip}{0pt}}
\setcounter{secnumdepth}{-\maxdimen} % remove section numbering
\usepackage{fancyhdr}
\pagestyle{fancy}
\fancyhf{}
\fancyhead[C]{\thepage}
\fancyfoot[L]{}
\fancyfoot[R]{}
\renewcommand{\headrulewidth}{0pt}
\renewcommand{\footrulewidth}{0pt}
\ifLuaTeX
  \usepackage{selnolig}  % disable illegal ligatures
\fi
\IfFileExists{bookmark.sty}{\usepackage{bookmark}}{\usepackage{hyperref}}
\IfFileExists{xurl.sty}{\usepackage{xurl}}{} % add URL line breaks if available
\urlstyle{same}
\hypersetup{
  pdftitle={Mis Notas},
  hidelinks,
  pdfcreator={LaTeX via pandoc}}

\title{Mis Notas}
\author{}
\date{\vspace{-2.5em}}

\begin{document}
\maketitle

\hypertarget{no-se}{%
\subsection{1 no se}\label{no-se}}

El tiempo X que utiliza un comercial para exponer un producto cuando LO
VENDE sigue, aproximadamente,

una distribución normal con parámetros mu =3

minutos 45 segundos y sigma = 10 segundos.

\begin{verbatim}
a. ¿Cuál es la probabilidad de que  consiga la venta  en menos de 4 minutos? 

b. ¿Y en más de 3.5 minutos?
\end{verbatim}

\hypertarget{section}{%
\subsection{2}\label{section}}

El tiempo X que utiliza un comercial para exponer un producto cuando NO
VENDE sigue, aproximadamente, una distribución normal con parámetros
mu=2 y sigma=0.8.

\begin{verbatim}
a. ¿Cuál  es el cuantil  0.95 de esta variable? Interpretarlo en el sentido de tiempo perdido por el comercial.

b. ¿Cuál es  el tiempo perdido   en el 40% de las llamadas más cortas?
\end{verbatim}

\hypertarget{section-1}{%
\subsection{3}\label{section-1}}

Un centro de atención telefónica por voz (call center) recibe por
termino medio 102 llamadas por hora. Suponed que el tiempo entre
llamadas consecutivas es exponencial.

\begin{verbatim}
a. Sea X el tiempo entre dos llamadas consecutivas ¿cuál es la distribución de X?

b. Calcular la probabilidad que pasen al menos 2.5 minutos hasta recibir la primera llamada.

c. Calcular la probabilidad que pasen menos de  3 minutos hasta recibir  la siguiente llamada.

d. Calcular la esperanza y la varianza de X.
\end{verbatim}

\hypertarget{section-2}{%
\subsection{4.}\label{section-2}}

Sea X una variable aleatoria normal con parámetros mu=1 y sigma=1.
Calculad el valor de b tal que P((X-1)\^{}2\textless= b)=0.1.

\hypertarget{section-3}{%
\subsection{5}\label{section-3}}

Sea Z una variable aleatoria N(0,1). Calcular P((Z-1/4)\^{}2
\textgreater1/16).

\hypertarget{section-4}{%
\subsection{6}\label{section-4}}

Un contratista de viviendas unifamiliares de lujo considera que el coste
en euros de una contrata habitual es una variables X que sigue una
distribución N(mu=600000,sigma=60000)

\begin{verbatim}
a.  ¿Cuál es la probabilidad de que  el coste del edificio esté entre 560000 y 660000 euros?

b. 0.2 es la probabilidad de que el coste de la vivienda supere ¿qué cantidad?

c. ¿Cuál es el coste mínimo del  5% de las casa más caras?
\end{verbatim}

\hypertarget{section-5}{%
\subsection{7}\label{section-5}}

Si X está distribuida uniformemente en (0,2) e Y es una variable
exponencial con parámetro lambda. Calcular el valor de lambda tal que
P(X\textless1)=P(Y\textless1).

\end{document}
